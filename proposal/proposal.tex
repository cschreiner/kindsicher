\documentclass[12pt] {article}
\usepackage[margin=1in]{geometry} %one inch margins
\usepackage{graphicx} %for figures
\usepackage{enumerate} %for lists
\usepackage{fancyhdr} %header
\pagestyle{fancy}
\usepackage[font={small,sf},format=plain,labelfont=bf,up]{caption}
\fancyhf{}
\fancyhead[l,lo]{\textit{Kindsicher: Safe-Browsing for Children at Home}} %left top header
\fancyhead[r,ro]{\thepage} %right top header

\usepackage{url}
\usepackage{indentfirst}

\begin{document}

\title{Kindsicher: Safe-Browsing for Children at Home}
\author{Chaitanya Achan, Philip Lundrigan, and Christian Schreiner}
\date \today
\maketitle
\setcounter{page}{1}

% x Problem definition and motivation. What are the goals of your project and
%   why are these goals important?
%
% x Why you chose that topic to work on and/or how the backgrounds of group
%   members contribute to the goal.
%
% x Any plans for how to divide the work that you have in place.
%
% x Your approach for addressing the problem that you defined in (1).
%
% x How you plan to evaluate your approach.
%
% - Main obstacles that you foresee along the way, and any contingency plans
%   that you have in place.
%
% x An initial discussion of the related work of which you are aware. If
%   appropriate, also explain how your approach is different from existing
%   approaches.
%
% x A list of milestones and (optionally) dates for those milestones.

\section*{Introduction}
It is important for parents to protect their children from purposeful or
accidental access to harmful websites on the Internet. Parent's can't ways be
watching what a child is doing and as a result, parental filtering is
necessary. The goal of this project is to create a controlled online
environment for children at home, even when the parents can not be closely
supervising them. The system will allow the children to only access known safe
Internet sites, without restricting parental access. The system will also log
all Internet traffic in and out of the home, and provide a means for parents to
view the logs, so they can make a rational response to any anomalies that
occur. The blocking of unapproved websites is the first line of defense in our system,
with logging acting as a second line of defense. It allows parents to review
Internet activity and change their approach if necessary.

We selected this project because all of us in the group have families with
children so it is applicable to all of us. After looking through different
options for parental filtering, we found no solution that matched our needs.
Combined we have experience with networking and the tools that we suggest to
use for this system.


\section*{Approach}

The first component of our system will control what computers children are
allowed to use. This is done by using the network login system,
SNISR~\cite{snisr}, to allow children to only log into ``children's'' computers
and not the parent's.  Parent's computers will have unrestricted access to the
Internet (though traffic from the parents' computers will be logged). The
system will allow the children's computers to only communicate with Internet
sites on a ``whitelist'', when communicating via TCP.  This will be implemented
by adding TCP connection shutdown rules to the SNORT network monitoring
tools~\cite{snort}. The system will include a whitelist of known safe Internet
sites.

The system will implement logging via the SNORT network monitoring tool, and
will make the logs available to the parents via SNORT's companion tool,
BASE~\cite{base}, previously called ACID LAB~\cite{acidlab}. The system will
not restrict any computer's access of sites via UDP, IMCP, or other non-TCP
protocols, though the system will still log all non-TCP traffic.

If time allows, we would like to make this system easy to use from a
non-technical user's perspective. There are three ``stretch goals'' we have
thought of. The first is allowing parents to add and delete sites from the
whitelist without requiring superuser access or system administrator
experience. Changes to the whitelist will be logged in a revision control
system. The second is to allow children to request that a site be added to the
whitelist (i.e. through a webpage). The parents will receive a notification of
the request, and can send a suitable authorization. The third improvement is to
develop a custom network monitoring report that makes more sense to a
non-technical parent than the standard monitor logs. A final stretch goal
unrelated to usability improvements is to do an independent security analysis
of SNISR. We do not want these goals to detract from building a system that is
safe and secure so we plan on addressing if there is time.

To evaluate our system, we plan on using it in a real family setting. We will
get feedback from the children and parents using the system as well as the
monitor logs to make sure the system is behaving as expected.

The project can be split up into three parts: integration with SNISR, blocking
non-whitelisted websites, and logging network usage. Each member of the group
with be in charge of one of these parts receiving help from the other group
members as needed.

We expect that a significant amount of the work on the project will be
developing an workable whitelist that allows children to do online homework and
other tasks. This could be a potential obstacle for the project. It is
understood that the list will need development and modification beyond the
scope of this project.


\section*{Thread Model}
The adversary of this system is the children using it. This includes children
trying to circumvent the system accidentally or purposefully. We assume that
children ages 5 to 16 will be using this system and have basic computer
knowledge.  We assume that children have physical access to the server that our
system will be running on.


\section*{Related Work}
There are many different commercial or free parental filters
(\cite{k9}\cite{netnanny}\cite{mcafee}) that already exist. These solutions
don't provide the technical details on how they work for comfortable use. It is
important for us to know how something like this works so we know the
limitations of it.

There are two major open source parental filters. The first,
KidLogger~\cite{kidlogger}, provides a wide variety of tracking tools such as
key logging, time tracking and web history monitoring. This software runs on
the child's computer which does not meet the needs of our project. The other
open source parental filter is DansGuardian~\cite{dansgaurdian}. DansGuardian
filters by content rather than website. Since we want control over what
websites children can visit and not just the content, we feel that this
approach does not fit our needs.  DansGuardian also has not been updated
recently (Sep 2012) and in order to be used, all traffic needs to be routed
through it. One of our design goals is to allow Internet use even during server
failure which DansGuardian does not meet.

\section*{Milestones}
\begin{itemize}
    \item Integration with SNISR
    \item Become familiar with tools (SNORT, BASE)
    \item Implement and demonstrate URL blocking
    \item Implement and demonstrate network logging tool
    \item Develop whitelist
    \item Integrate system into family environment
\end{itemize}

{
  \bibliographystyle{acm}
  \bibliography{biblio}
}

\end{document}
