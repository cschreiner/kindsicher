\documentclass[12pt] {article}
\usepackage[margin=1in]{geometry} %one inch margins
\usepackage{graphicx} %for figures
\usepackage{enumerate} %for lists
\usepackage{fancyhdr} %header
\pagestyle{fancy}
\usepackage[font={small,sf},format=plain,labelfont=bf,up]{caption}
\fancyhf{}
\fancyhead[l,lo]{\textit{ Safe-Browsing for Children at Home}} %left top header
\fancyhead[r,ro]{\thepage} %right top header

\usepackage{url}

\begin{document}

\title{Safe-Browsing for Children at Home}
\author{Chaitanya Achan, Philip Lundrigan, and Christian Schreiner}
\date \today
\maketitle
\setcounter{page}{1}

\section*{Project}

The goal of this project is to create a controlled online environment for
children at home, even when the parents can not be closely supervising them.
The system will allow the children to only access known safe internet sites,
without restricting parental access.  The system will also log all
internet traffic in and out of the home, and provide a means for parents to
view the logs, so they can make a rational response to any anomalies that
occur.

Top level requirements:

The system will allow the children to only log into ``children's'' computers,
not the parents'.  This will be implemented by adding a feature to the SNISR~\cite{snisr}
network login system.

The system will allow the parents' computers unrestricted access to the
internet (though even traffic from the parents' computers will be logged).

The system will allow the childrens' computers to only communicate with
internet sites on a ``whitelist'', when communicating via TCP.  This will be
implemented by adding TCP connection shutdown rules to the SNORT network
monitoring tools.

The system will include a whitelist of known safe internet sites.  We (the
authors of this proposal) expect that a significant amount of the work on
the project will be developing an workable whitelist that allows children to
do online homework and other tasks.  It is understood that the list will
need development and modification beyond the scope of this project.

The system will implement logging via the SNORT network monitoring tool, and
will make the logs available to the parents via SNORT's companion tool,
BASE.  (Note: BASE's predecessor was ACID LAB.)

The system will not restrict any computer's access of sites via UDP, IMCP,
or other non-TCP protocols, though the system will still log all non-TCP
traffic.

Stretch requirement: The system will allow a parent to add or delete sites
from the whitelist. without requiring superuser access or system
administrator experience.  Changes to the whitelist shall be logged in a
revision control system.

Stretch requirement: The system will allow the children to request that a
site be added to the whitelist by entering its URL onto a webpage.  The
parents will be able to receive such a notification remotely, and send a
suitable authorization to the system.  We (the authors of this proposal)
anticipate implementing this by sending an email to the parents that is
accessible via smartphone, and to accept a reply via email.  The system
would need to only accept replies from the parents' email addresses, and may
need additional guards against spoofing.

Stretch requirement: Do an independent security analysis of SNISR.

Stretch requirement: If needed and desired: develop a custom network
monitoring report that makes more sense to a non-technical parent than the
standard reports.

Snort\cite{snort}

\section*{Thread Model}

\section*{Related Work}

\section*{Milestones}

URLS:
For BASE: http://sourceforge.net/projects/secureideas/
        and http://base.professionallyevil.com/
For ACID LAB: http://acidlab.sourceforge.net

{
  \bibliographystyle{acm}
  \bibliography{biblio}
}

\end{document}
