% primary responsibility: CAS+Phil

\section{Related Work}

Protecting children from the potential harms of the Internet has been studied
from a psychological point of view~\cite{ybarra2005exposure, ho_statistical}.
Guidelines have been developed to help parents know how to expose their
children to the Internet. For example, the EU published an article on this
subject~\cite{holloway2013zero, livingstone2010risks}. Other studies have
looked at to see if such guidelines are effective or
not~\cite{livingstone2008parental}. These studies show that there is a need for
Internet filtering in homes to help protect children. These studies differ from
our work in that they try to understand and measure the effect of the Internet
on children's lives, whereas we look at building a filtering system.

There have also been studies on using firewalls to control Internet
access~\cite{ivanovic, nguyen}. These approaches take a similar approach to
our system, but in our system, we don't use a firewall or any hardware specific
device.

A important part of our system is defining a correct whitelist. Whitelist
implementations have been looked at to prevent specific web-based
attacks~\cite{han_automated_whitelist, iha_implementation}. These whitelists
were used for different purposes than what we are using our whitelist for.

Finally, prior work also includes web-page classification
mechanisms~\cite{baykan_et_al_url_based_classification,
chen_et_al_novel_web_page_filtering, ho_statistical} with the intent of
identifying content and controlling access. We decided to avoid such techniques
because they are hard to maintain and easily tricked. In our system, we do not
classify webpages. Instead we block pages if they are not on the whitelist.
This gives parents complete control over the system and as a result, parents
are not dependent to the whim of the classification algorithm.
