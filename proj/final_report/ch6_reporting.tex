\subsection{Reporting}

The second goal of this project is to provide parents with the ability to
monitor and derive reports on their child's Internet access. 
%
We looked at incorporating BASE (Basic Analysis and Security Engine) into Kindsicher to
render this functionality. 
%
BASE is based of the ACID (Analysis Console for Intrusion Databases) project
and provides a web-based front-end to query and analyze alerts generated by
the SNORT IDS system.

We first define a rule within SNORT to capture all HTTP traffic originating
from the home network.  

\verb+alert tcp any any <> any 80 (msg:"HTTP Alert")+ 

This causes SNORT to log all HTTP requests from any source IP and port to any
destination IP and port 80 (indicating HTTP) to a MySQL database. 
%
BASE then reads the alert information from this database and displays it in a
web-browser. (Figure R1)

\begin{figure}
\includegraphics{figures/R1_BASE_Flow}
\end{figure}
% TODO: do something to label this R1 or similar
% TODO: add caption "BASE Data Flow"

To access base you need to open a browser window and type in “<base server
name>/base” in the URL field.
%
Figure R2 shows the BASE home page presented to an user on logging into BASE.

\begin{figure}
\includegraphics{figures/R2_BASE_Main}
\end{figure}
% TODO: do something to label this R2 or similar
% TODO: add caption "BASE Home Page"
% TODO: get these graphics more in-line interspersed with the text

For this project we were particularly interested in destination IP
addresses. 
%
On clicking on the count of destination IP addresses shown
in Figure R2, BASE displays a list of all the IP addresses accessed
from the home network (Figure R3), along with other associated
information such as the number of times the IP was accessed and the
number of source addresses that accessed it.

\begin{figure}
\includegraphics{figures/R3_BASE_IPList}
\end{figure}
% TODO: do something to label this R3 or similar
% TODO: add caption "List of destination IPs accessed from the network"
% TODO: get these graphics more in-line interspersed with the text

BASE also provides several canned reports to graphically represent
network data. Figure R4 shows an example of report selection and
parameter specification.

\begin{figure}
\includegraphics{figures/R4_BASE_Report}
\end{figure}
% TODO: do something to label this R4 or similar
% TODO: add caption "BASE report selection and initiation
% TODO: get these graphics more in-line interspersed with the text

Based on our study of BASE we came up with the following conclusions.

\textbf{Advantages of BASE:}
\begin{itemize}

\item It is built to work in conjunction with SNORT.

\item It provides a simple web-interface for users to view and
  analyze network traffic.

\item It comes with several in-built reports to help in the analysis.

\end{itemize}

\textbf{Disadvantages of BASE:}
\begin{itemize}

\item It deals with only IP addresses and does not display the
  corresponding domain names. 
  % 
  IP addresses are not intuitive to users and this raises a real
  concern given the fact that our target users are parents who are not
  necessarily tech-savvy.

\item It also displays only the number of times a site has been
  accessed and gives no other information like the date and time of
  access.

\item The in-built reports do not support any correlation of which
  source IP(s) accessed which destination IP(s), reducing a parent's
  understanding of their child's Internet access.

\end{itemize}

Although BASE does provide an avenue for uses to view and analyze web
access, we could say it only partially meets the reporting goals for
Kindsicher because of these disadvantages. 
%
This is further detailed in the Future Work/Improvements section.
