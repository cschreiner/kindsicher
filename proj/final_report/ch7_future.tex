% primary responsibility: Chaitu

\section{Future Work}

As described in the blocking section, we encountered some latency issues with
Snort that inhibited its blocking ability under certain conditions. For
Kindsicher to be effective in all circumstances, we need to investigate Snort
further. This might involve modifying the source of Snort to remove all extra
functionality. This could also involve writing our own program that blocks
using TCP Resets.

In the Reporting section, we described some of the shortcomings of BASE,
primary its reporting on network access using only IP addresses.  The
reporting can be made more user friendly by incorporating domain names instead
of or in addition to IP addresses. It could also be modified to provide a
richer set of data to parents, that makes understanding their child's web
access footprint much more intuitive.

Over the longer term, we also need a process that automates maintaining the
addresses in the Snort white-list. We envision that a child can request
addition of a new website domain as a text message to the parent's phone and
the parent could then chose to approve or deny that request, automatically
updating the white-list as needed. This would allow parents to change the
whitelist with minimal effort.
