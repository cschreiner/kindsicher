% primary responsibility: CAS+Phil

\section{Design}

The goal of our system is two fold: block websites that are not part of the
white list and to record network statistics about the Internet usage of the
family. The second goal is important because it is not enough to just block
unwanted websites, health Internet usage habits must be developed as well. By
logging Internet usage, parents can identify any unhealthy behaviors with the
use of the Internet. For example, say a teenager is staying up late to use the
Internet and the website they are viewing is on the whitelist. The content the
teenager is viewing might not be bad (such as a simple Internet game), but
staying up late is not a healthy habit. With this information, the parents are
able to address this problem in a direct manner.

For both blocking and reporting, we used the intrusion detection system (IDS),
Snort~\cite{snort}. An IDS is a piece of software that sits on individual
clients (called host based IDS or HIDS) or on a network (called network based
IDS or NIDS). It monitors traffic that passes through it and detects if any
type of attack common is occurring. It records this information in a database
and notifies an network administrator if it detects any serious problems.

Snort is rule based. This means that a user can write rules that if the rule
conditions are met, Snort performs an action. Figure XXX shows the general
syntax of a rule. There are communities of Snort users that write rules for
common attacks and network problems so that typical users do not need a deep
knowledge of intrusion detection to get benefit from Snort.

% TODO: Describe Snort rule types
% TODO: Describe network topology

How Snort was used to accomplish each goal (blocking and reporting) is outlined
in the sections below.

