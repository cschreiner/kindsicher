% primary responsibility: CAS+Phil

\section{Design}




\subsection{Login}

The TCP/IP protocol, and the HTTP protocol that rides on top of it, identify
the sending computer, \emph{NOT} the user name on that computer.  
% 
Since Kindsicher's goal is to restrict access for children, not for adults,
the only way to distinguish between children's internet access and adults'
access is to designate certain computers for children's use only, and other
computers for adults' use only.  
%
Thus, there needs to be some mechanism to ensure that the children can only
log into hosts designated for children.

The network we used for testing uses SNISR 
%
\footnote{ SNISR is a network login and directory information system, akin to
LDAP, NIS, or Hesiod.  Its signature advantage is that all clients have a
local record of account and group information at all times, so mobile clients
can be disconnected from the network and reconnected at will.  For more
information, see http://www.mathoni.net/cas/swforge/snisr for more
information. }
%
for login directory services.  
%
The current released version, SNISR 1.1, allows any user to log in on any host.  
%
Part of this project, therefore, was upgrading SNISR to allow the network
administrator to specify classes of hosts that only certain users could log
into.
%
SNISR 2.0's baseline ALPHA_6 contains this feature, and is expected to begin
beta testing soon.


