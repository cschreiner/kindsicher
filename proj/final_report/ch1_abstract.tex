In a modern society, children must increasingly use the Internet for required
tasks such as homework and communication with parents and other family
members, not just for optional tasks (such as games and entertainment). 
%
The amount of required use means that children must often use the Internet
when their parents cannot supervise them. 
%
This carries significant risk that the children will encounter Internet
content they are not able to handle, or that would exploit them. 
%
Most previous work on improving children's Internet safety focuses on
identifying ``bad'' content and blocking it. 
%
The amount of work to identify ``bad'' content is, however, beyond the
capabilities of most families. 
%
Commercial services exist that classify content, but they are still stretched
to adequately deal with the volume of legitimate content, and black hats are
continually finding new means of circumventing the services' products. 
%
Further, outsourcing content classification prevents a parent from restricting
content for family-specific reasons. 
%
For example, a child may have a phobia about spiders, or a parent may want to
insist on accompanying a child whenever they visit certain online shopping
sites, so the parent can teach good consumer practices during the shopping
session. 

We present Kindsicher, a children's Internet safety system that addresses
these issues by taking the opposite approach: parents define the Internet
sites they believe are appropriate for their children to visit unsupervised,
and access to other sites requires parental intervention. 
%
Kindsicher has the additional advantage that it is implemented as home network
infrastructure, so its protection is automatically extended to new devices
brought into the home (for example, when a friend arrives with a Wi-Fi
tablet). 

