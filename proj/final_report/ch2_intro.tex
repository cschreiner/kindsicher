% primary responsibility: CAS

\section{Introduction}

In a modern society, children must increasingly use the Internet for required
tasks such as homework and communication with parents and other family
members, not just for optional tasks (such as games and entertainment). 
%
The amount of required use means that children must often use the Internet
when their parents cannot continually supervise them. 
%
This carries significant risk that the children will encounter Internet
content they are not able to handle, or that would exploit them. 
%
Most previous work on improving children's Internet safety focuses on
identifying ``bad'' content and blocking it. Some of this work relies
on autoclassification \{TODO: add link to DansGuardian\}, some on
human classification \{TODO: add link to OpenDNS\}, and some on a
combination \{TODO: add link to NetNanny\}.
%
The amount of work to identify ``bad'' content is, however, beyond the
capabilities of most families.
%
Commercial services exist that classify content, notably OpenDNS
\{TODO: add link\} but they are still stretched to adequately deal
with the volume of legitimate content, and black hats are continually
finding new means of circumventing the services' products.  
%
Furthermore, the trend in commercial product literature is to focus
features and testimonials, and omit technical details.  \{TODO: add links to OpenDNS, and some other product websites.\}
%
This prevents parents from making a rational comparison between
products and evaluate potential for ineffectiveness or circumvention
without purchasing each and performing detailed technical evaluations.
%
Further, outsourcing content classification prevents a parent from restricting
content for family-specific reasons. 
%
For example, a child may have a phobia about spiders, or a parent may want to
insist on accompanying a child whenever they visit certain online shopping
sites, so the parent can teach good consumer practices during the shopping
session. 

This paper presents Kindsicher, a children's Internet safety system
that addresses these issues by taking the opposite approach: parents
define the Internet sites they believe are appropriate for their
children to visit unsupervised, and access to other sites requires
parental intervention.
%
Kindsicher offers the following advantages:

\{TODO: order these for maximum rhetorical effect \}
\begin{itemize}

\item Kindsicher is completely open-source software, so the
  technically inclined are free to examine it, improve it if desired,
  and report their results to others.

\item Kindsicher is blocks traffic by interrupting TCP connections, as
  opposed to blocking DNS name lookups, so it cannot be circumvented
  by hard-coding an IP address in a URL (e.g. http://155.98.65.24/),
  which is a common black-hat tactic.

\item Kindsicher is completely implemented as home network
  infrastructure, so its protection is automatically extended to new
  devices brought into the home (for example, when a friend arrives
  with a Wi-Fi tablet). Most other solutions have a component that
  must be installed on the client device.

\item Because Kindsicher is completely implemented on the home
  network, it does not slow network performance while content is
  routed via a cloud-based filtering server.  This can be a factor for
  current and modern rich-content webpages.

\{TODO: anything else?\}

\begin{itemize}

\{TODO: should we do a comparison with other work to make the point that Kindsicher fills a gap in the arsenal of tools available to parents?\}
