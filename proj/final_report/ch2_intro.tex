% primary responsibility: CAS


% force the word INTRODUCTION to not appear by itself at the end of a column
\vspace{20mm}

\section{Introduction}
\nopagebreak
In a modern society, children must increasingly use the Internet for required
tasks such as homework and communication with parents and other family
members, not just for optional tasks (such as games and entertainment).
%
The amount of required use means that children must often use the Internet
when their parents cannot continually supervise them.
%
This carries significant risk that the children will encounter Internet
content they are not able to handle, or that would exploit them.

Most previous work on improving children's Internet safety focuses on
identifying ``bad'' content and blocking it. Some of this work relies
on autoclassification (e.g. Dans Guardian~\cite{dansguardian}), some on
human classification (e.g. OpenDNS~\cite{opendns}), and some on a
combination (e.g. NetNanny~\cite{netnanny}).
%
The amount of work to identify ``bad'' content is, however, beyond the
capabilities of most families.

Commercial services exist that classify content~\cite{opendns, netnanny,
mcafee,k9}, but they are still stretched to adequately deal with the volume of
legitimate content, and black hats are continually finding new means of
circumventing the services' products.
%
It is disappointing to observe a pattern in commercial product literature of
focusing on features and testimonials, and omitting technical
details.~\cite{opendns, kidlogger, mcafee, k9}
%
This prevents parents (or a technically inclined friend of the parents) from
making a rational comparison between products and evaluating potential for
ineffectiveness or circumvention without purchasing each product and
performing detailed technical experiments.
%
Further, outsourcing content classification prevents a parent from restricting
content for family-specific reasons.
%
For example, a child may have a phobia about spiders, or a parent may want to
insist on accompanying a child whenever they visit certain online shopping
sites, so the parent can teach good consumer practices during the shopping
session.

This paper presents Kindsicher\footnote{Kindsicher is a portmanteau of the
German words for ``child'' and ``safe.''}, a children's Internet safety system
that addresses these issues by taking the opposite approach:
%
parents define the Internet sites they believe are appropriate for their
children to visit unsupervised, and access to other sites requires parental
intervention.
%
Kindsicher offers the following advantages:

\begin{itemize}

\item Kindsicher allows the parents full control over the Internet sites to be
designated as ``safe''.
%
Kindsicher's architecture allows for parents to define exceptions to the
approved site list so children can browse additional content when parents
are present.
%
It would be conceptually simple to extend this so parents could temporarily
add sites via a web-capable smartphone, in case children needed to access
another site while their parents were away.

\item Kindsicher is completely implemented as part of a home network
infrastructure, so its protection is automatically extended to new
devices brought into the home (for example, when a friend arrives
with a Wi-Fi tablet). Most other solutions have a component that
must be installed on the client device.

\item Since Kindsicher is completely implemented on the home
network, it does not slow network performance while content is
routed via a cloud-based filtering server.  This can be a factor for
current and modern rich-content web pages.

\item Kindsicher logs all Internet traffic, so parents can have a rational
conversation with their children if a problem occurs.
%
(Problems might range from a teenager excessively playing online games, to a
child coming to parents in tears because something seen on the Internet
scared him/her.)

\item Kindsicher's logging feature means that if black hat malware tried to
attack the Kindsicher server, the attack would be logged, so technically
competent humans could address the problem.
%
Even a non-technically inclined parent could still see which computer in the
home was involved in the attack (either as a source or as a victim), and take
that computer to a repair shop for malware removal and an anti-virus update.

\item Kindsicher blocks traffic by interrupting TCP connections, as
opposed to blocking DNS name lookups, so it cannot be circumvented
by hard-coding an IP address in a URL (e.g. http://155.98.65.24/),
which is a common black-hat tactic.

\item Kindsicher is completely open-source software, so the
technically inclined are free to examine it, improve it if desired,
and share their results with others.

\end{itemize}

No product or system exists today that has all of these features. We hope that by developing Kindsicher, we can show that each of these features is important and should be integrated into previous systems.
